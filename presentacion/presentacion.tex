\documentclass{beamer}

\usepackage[utf8]{inputenc}

\title[Fono Estafa]{Fono Estafa: lista negra de números teléfonos}
\author[Rodrigo Ahumada M.]{Equipo indignados, Rodrigo Ahumada M., Chile}
\date[]{3 diciembre 2011}

\begin{document}


\frame{\titlepage}

\section*{Contenido}
\frame
{
	\frametitle{Lista de contenidos}
	\tableofcontents
}


\section{Problemática}
\frame
{
	\frametitle{Problema}
	
	Muchos de los intentos de estafas telefónicas son fácilmente detectables, y sólo
	falta unir la información para alertar a posibles víctimas.
}


\section{Propuesta}
\frame
{
	\frametitle{Solución propuesta}

	Crear un centro de denuncias de números telefónicos involucrados en estafas,
	que permita a las personas más informadas del tema advertir a las menos informadas,
	y así dificultar el trabajo de los estafadores.
}


\frame
{
	\frametitle{Partes de la aplicación}

	\begin{itemize}
	\pause
	\item Página web con base de datos de denuncias
	\pause
	\item Aplicación para móviles que permite denunciar números telefónicos
	\pause
	\item Aplicación para móviles que intercepta llamadas entrantes y consulta a la base de datos
	\end{itemize}
}


\section{Como soluciona el problema}
\frame {
	\frametitle {Como ayuda a evitar estafas}

	\begin{itemize}
	\pause
	\item Un estafador trata de estafarme
	\pause
	\item Yo denuncio el número de teléfono
	\pause
	\item El numero queda registrado en lista negra, por un tiempo (1 semana)
	\pause
	\item Mi tía recibe la llamada del estafador (usando el mismo número)
	\pause
	\item antes de que alcance a contestar, la aplicacion móvil a consultado por el número
	\pause
	\item Mi tía ve el aviso de posible intento de estafa
	\pause
	\item Mi tía queda iluminada
	\end{itemize}
}

\section{Proyecciones}
\frame
{
	\frametitle{Proyección}

	\begin{itemize}
	\pause
	\item Primera vez que trabajo con Android
	\pause
	\item Este servicio deberia ser dado por las compañías telefónicas
	\end{itemize}
}

\frame
{
	FIN
}

\end{document}
